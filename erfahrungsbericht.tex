\documentclass[12pt, a4paper]{article}

\usepackage[ngerman]{babel}
\usepackage[T1]{fontenc}
\usepackage[utf8]{inputenc}
\usepackage{tgheros}
\renewcommand*\familydefault{\sfdefault}
\usepackage{microtype}

\usepackage{amssymb}
\usepackage{tikz}

\newcommand{\heading}[1]{\bigskip\noindent\textbf{\MakeUppercase{#1}:}\medskip}

% \newcommand{\blank}{\rule{4cm}{0.1pt}}
\newcommand{\blank}{\hrulefill}

\newcommand{\yesnoboxes}{\hspace{1ex}\(\square\) ja\hspace{5ex}\(\square\) nein}
\newcommand{\yesboxes}{\hspace{1ex}\(\boxtimes\) ja\hspace{5ex}\(\square\) nein}
\newcommand{\noboxes}{\hspace{1ex}\(\square\) ja\hspace{5ex}\(\boxtimes\) nein}


\usepackage{enumitem}
\setlist[description]{%
  font=\normalfont\itshape,
  topsep = \medskipamount,
  labelindent = 1em}
\newlist{checklist}{itemize}{2}
\setlist[checklist]{label=\(\square\)}


\begin{document}

\let\oldindent\parindent
\setlength{\parindent}{0pt}
\setlength{\parskip}{\bigskipamount}

\begin{tikzpicture}[remember picture, overlay]
  \node [xshift=-5.7cm, yshift=-2cm]  %, minimum width=3cm, minimum height=3cm, draw
  at (current page.north east)
  {\includegraphics[width=4cm]{logo}};
\end{tikzpicture}
\vspace{-1.5cm}

% Bericht soll anderen Studierenden, die sich für einen Auslandsaufenthalt interessieren, zugänglich
% gemacht werden und auf der Homepage des Büros für Internationale Beziehungen und Mobilitätsprogramme
% veröffentlicht mind. 1-2 Seiten Abgabe in ausgedruckter und elektronischer Form

\begin{centering}
  \large \textbf{ERFAHRUNGSBERICHT}
  \vspace{\baselineskip}
  \hrule
  \vspace{\baselineskip}
\end{centering}


\heading{Persönliche Daten}
\begin{description}
% \item[Name des/der Studierenden:] (freiwillige Angabe) \blank
% \item[Matrikelnummer:] (freiwillige Angabe) \blank
% \item[E-Mail:] (freiwillige Angabe) \blank
\item[Gastinstitution und Angabe der Fakultät:] \blank
\item[Studienrichtung an der TU Graz:]  \blank
\item[Aufenthalt an der Gastinstitution:]  \blank
\item[Mobilitätsprogramm (zB. Erasmus+, ISEP, Joint Study):]  \blank
\end{description}


\heading{Allgemeines}

Wie hoch waren die monatlichen Kosten für\ldots?

\begin{description}
\item[Unterbringung:]  \blank
\item[Verpflegung:] \blank
\item[Fahrtkosten:]  \blank
\item[Kopien, Skripten, etc.:]  \blank
\item[Sonstiges:]  \blank
\end{description}


\heading{Sprachkurs (\MakeLowercase{falls zutreffend})}

Haben Sie vor beginn des Aufenthalts einen vorbereitenden Sprachkurs absolviert?
\begin{description}
\item[In Österreich:] \yesnoboxes
\item[Im Gastland:] \yesnoboxes
\end{description}
Wenn ja:
\begin{description}
\item[Wo/bei welcher Einrichtung?] \hrulefill
\item[Wie zufrieden waren Sie?] \hrulefill
\item[Kosten des Sprachkurses:] \hrulefill
\item[Wurde der Sprachkurs an der TU Graz anerkannt?] \yesnoboxes
\end{description}


\heading{Unterkunft}

Wie haben sie Ihre Unterkunft gefunden?

\begin{checklist}
\item Selbst gesucht
\item Universität hat Unterkunftssuche übernommen
\item Freunde
\item Anderes: \blank
\end{checklist}

Wie zufrieden waren Sie? \blank 

Tipps für NachfolgerInnen: \blank
% (z.B. bestimmte Zeitungen, Homepages mit Wohnungsangeboten; wichtige länderspezifische Hinweise;
% Preis-Leistungsverhältnis, etc.)


\heading{Fragen zum Studium}

In welchem Maße waren universitäre Einrichtungen vorhanden bzw. zugänglich?
\begin{description}
\item[Computerräume:] \blank
\item[Labors, Zeichensäle, etc.:] \blank
\item[Andere Einrichtungen:] \blank
\end{description}

Hatten Sie Probleme bei der \underline{Voraus}anerkennung der Lehrveranstaltungen an der TU Graz?
\hspace{\oldindent}\yesnoboxes \\
Wenn ja, welche? \blank

Wurden nach Ihrer Rückkehr alle Lehrveranstaltungen für das Studium an der TU Graz anerkannt?
\yesnoboxes \\
Wenn nein, welche nicht und aus welchem Grund? \blank

Tipps für NachfolgerInnen: \blank
% z.B. Informationen zu Lehrveranstaltungen, Prüfungen, etc.

Wird sich Ihr Studienfortgang an der TU Graz als Folge Ihres Auslandsaufenthalts verzögern?
\yesnoboxes


\heading{Erfahrungsbericht}

%  (mind. 1 – 2 Seiten)
% - Gesamteindruck zum Auslandsaufenthalt mit Bezug auf den fachlichen Nutzen und die generellen Erfahrungen vor Ort 
% - Reisevorbereitungen, Details zur Visumsantragstellung, Versicherungen, Ankunft am Studienort (z.B. Transport Flughafen - Stadtzentrum), Abreise, etc. 
% - Betreuung an der Einrichtung, z.B. durch die Lehrenden 
% - Studierendenleben am Studienort (Auto von Vorteil?, Bücherkosten, Arbeiten am Campus möglich?, etc.) 
% - ev. Angabe einer eigenen Homepage, auf der sich weitere Infos / Erfahrungen zum Aufenthalt befinden 
% - ev. Fotos      

\end{document}
